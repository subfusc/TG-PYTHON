\documentclass[onesided,norsk,11pt]{beamer}
\usepackage[utf8]{inputenc}
\usepackage{babel}
\usepackage{listings}
\usebackgroundtemplate{\includegraphics[width=\paperwidth]{UiO_Campus_TG1024.png}}

\lstset{
  basicstyle=\small
}

\begin{document}

\begin{frame}
  REGULÆRE UTTRYKK I PYTHON
\end{frame}

\begin{frame}
  \frametitle{Hvordan få regulære uttrykk inn i python}
  For å gjøre et søk etter noe i en tekst, så må vi bruke
  de følgende to linjene.
  \lstinputlisting[language=Python]{re1.py}
\end{frame}

\begin{frame}
  Du kan søke etter hvilken som helst tekststreng.
  \lstinputlisting[language=Python]{re2_1.py}
\end{frame}

\begin{frame}
  Og vi kan finne ut om vi har funnet noe ved å si:
  \lstinputlisting[language=Python]{re2_2.py}
\end{frame}

\begin{frame}
  \begin{itemize}
    \item Bortsett fra å kunne søke etter noe i en tekst, 
      hva er spessiellt med regulære uttrykk?
    \item Spessielle symboler i teksten vil gjøre søket
      mer avansert.
  \end{itemize}
\end{frame}

\begin{frame}
  \frametitle{Symbolet . ( \textbackslash~\textbackslash~. )}
  . kan være hva som helst. Hvis du ville ha en dot (for eksempel
  søke etter filer som ``hei.txt''), må du bruke \textbackslash~\textbackslash~.
  \lstinputlisting[language=Python]{redot.py}
\end{frame}

\begin{frame}
  \frametitle{Symbolet *}
  * vil finne alle symbol som er rett foran den 0 eller flere ganger.
  Eksempel: ``hu*nd'' vil matche
  \begin{itemize}
    \item hnd
    \item hund
    \item huund
    \item huuund
    \item etc.
    \end{itemize}
\end{frame}

\begin{frame}
  \frametitle{Paranteser ()}
  Med () kan du lage større grupper av symboler og få ut hva
  du har funnet. 
  Eksempel: ``h(..)d''
  vil matche hund og i resultatet fra søket kan du finne ut at
  .. var ``un''
\end{frame}

\begin{frame}
  Med dette kan vi begynne å crawle Webben for info.
  For eksempel, hvordan vil du prøve å finne bildet i
  dette dokument?
  \lstinputlisting[language=html]{eksempel2.html}
\end{frame}

\begin{frame}
  Vi vet at bilder må starte med ``\textless~img'' og slutte med
  ``\textgreater~''. Hva som er inni der er uvisst/
  Så om vi ikke vet hva som er der, så kan vi bruke ``.'', som
  representerer hvilken som helst bokstav. Men vi trenger den
  mer enn 1 gang. Hvordan løser vi det problemet?
  ``*'' vil ta et symbol mer enn 1 gang. ``.*'' vil da ta hva 
  som helst.
  Så da prøver vi '\textless~img.*src=''.*''.*~\textgreater'
  Da vil du kunne finne ut om et dokument har et bilde, men for å
  hente det ut, må vi bruke ().
  '\textless~img src=''(.*)'' \textgreater'
\end{frame}

\begin{frame}
  \frametitle{Problem}
  Et problem med denne koden
  \lstinputlisting[language=python]{rebilde.py}
\end{frame}

\begin{frame}
  \frametitle{Symbolet ?}
  
  ? kan bety to ting
  \begin{itemize}
    \item sammen med * betyr det at du vil prøve å finne 
      så lite som mulig med denne stjernen.
    \item sammen med alt annet, betyr at det rett forran
      kan forekomme 0 eller 1 gang
    \end{itemize}

    Det er veldig greit om du skal unngå å få feil ting tilbake.
\end{frame}

\begin{frame}
Så hvis vi prøver igjen, der vi pakker ? på alle stjernene våre, 
slik at vi leter etter så lite som mulig hver omgang.
Da får vi '\textless~img.*?src=''(.*?)''.*?\textgreater~'
la oss prøve
\end{frame}

\begin{frame}
Prøv på en live webside ved bruk av urllib 
og urlopen.
\end{frame}

\end{document}